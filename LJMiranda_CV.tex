%% start of file `template.tex'.
%% Copyright 2006-2015 Xavier Danaux (xdanaux@gmail.com).
%
% This work may be distributed and/or modified under the
% conditions of the LaTeX Project Public License version 1.3c,
% available at http://www.latex-project.org/lppl/.


\documentclass[12pt,a4paper]{moderncv}


% moderncv themes
\moderncvstyle{banking}
\moderncvcolor{black}

\usepackage[utf8]{inputenc}                     % set encoding
\usepackage[scale=0.75]{geometry}               % customize paper
\usepackage[unicode]{hyperref}                  % to use hyperlinks
\usepackage{xcolor}                             % better colors
\usepackage{color}                              % syntactic sugar
\usepackage{natbib}
\usepackage[T1]{fontenc}
% \usepackage{charter}
\usepackage{libertine}
%\usepackage{libertinust1math}
% \usepackage[charter]{mathdesign}
\usepackage[scale=0.875]{sourcecodepro}
%\usepackage[lf]{FiraSans}

% Setup bib formatting
\makeatletter\renewcommand*{\bibliographyitemlabel}{\@biblabel{\arabic{enumiv}}}\makeatother

% Setup fonts
\renewcommand*{\namefont}{\fontsize{24}{29}\mdseries\upshape}
%\renewcommand{\familydefault}{\sfdefault}

% Setup bullet points
\renewcommand{\labelitemi}{\scriptsize\color{black}{$\bullet$}}

\renewcommand{\bibpreamble}{
    You can also check my {\color{blue}\httplink[Google 
    Scholar]{scholar.google.com/citations?user=2RtnNKEAAAAJ&hl=en}}
    profile. Note: an asterisk (*) denotes equal contributions.
}

% PERSONAL DATA --------------------------------------------------------
\firstname{Lester James V.}
\familyname{Miranda}
% \address{Metro Manila, Philippines}
\email{ljvmiranda@gmail.com}
\homepage{ljvmiranda921.github.io}
\social[github]{ljvmiranda921}
\extrainfo{Last updated: October 30, 2023} % \today

\title{CV}

\begin{document}

\maketitle

\textsc{Summary}: Lj Miranda specializes in natural language processing with over five years of
experience in consulting, open-source software development, and research. 

% EXPERIENCE -----------------------------------------------------------
\section{Experience}

\cventry{Oct 2023 -- present}
{Predoctoral Young Investigator}
{Allen Institute for AI}
{Seattle, US}{AllenNLP Team}
{
    \begin{itemize}
        \item As a {\color{blue}\httplink[predoctoral
        researcher]{allenai.org/predoctoral-young-investigators}}, conducts
        cutting-edge research in the field of natural language processing (NLP) and computational linguistics.
        % Research areas: efficient NLP and corpus linguistics.
    \end{itemize}
}
\vspace{0.5em}

\cventry{Oct 2021 -- July 2023}
{Machine Learning Engineer}
{ExplosionAI GmbH}
{Berlin, DE}{spaCy Team}
{
    \begin{itemize}
        \item Authored spaCy's first technical paper, \textit{Multi-hash
        embeddings in spaCy}, that benchmarks the library's hash-embedding trick
        (first-author).
        \item Developed annotation workflows for a data annotation product,
        {\color{blue}\httplink[Prodigy]{prodi.gy}}, that integrates large
        language models (LLM) like GPT-3.5/4 to common natural language processing tasks.
        \item Improved spaCy's sequence labeling component,
        {\color{blue}\httplink[Span Categorizer]{explosion.ai/blog/spancat}}, by
        adding new features, writing documentation, performing benchmark
        experiments, and optimizing performance.
        \item Co-developed several open-source software NLP libraries and
        developer tools including 
        {\color{blue}\httplink[spacy-llm]{github.com/explosion/spacy-llm}}
        (production LLM pipelines),
        {\color{blue}\httplink[vscode-prodigy]{github.com/explosion/vscode-prodigy}}
        (Visual Studio Code extension for data annotation), and
        {\color{blue}\httplink[spaCy projects]{github.com/explosion/projects}}
        (end-to-end NLP workflows for production).
    \end{itemize}
}
\vspace{0.5em}

\cventry{Oct 2018 -- Jul 2021}
{Machine Learning Researcher}
{Thinking Machines Data Science, Inc.}
{Metro Manila, PH}{Machine Learning Team}
{
    \begin{itemize}
        \item Developed several production-grade natural language processing
        applications for a major investment firm in Singapore, ranging from
        in-house search engines to document processing tools.
        \item As Tech Lead, led a project team to deliver a {\color{blue}
        \httplink[large-scale digitization
        project]{stories.thinkingmachin.es/doc-ai-world-bank-case-study/}} of
        all the local governments' financial statements across the country for
        The World Bank.
        \item Automated {\color{blue}\httplink[building detection from aerial
        images]{stories.thinkingmachin.es/wealth-detection-satellite-image/}}
        for one of the largest telecommunications companies in the Philippines
        using computer vision techniques.
    \end{itemize}
}
\vspace{0.5em}

% INTERNSHIPS -----------------------------------------------------------
\subsection{Internships}
\cventry{Aug 2018 -- Sep 2018}
{Research Intern}
{Preferred Networks, Inc.}
{Tokyo, JP}{ChainerRL Team}
{
    \begin{itemize}
        \item Developed a reinforcement learning parallelization framework
            based on batch Proximal Policy Optimization (PPO) for the
            open-source
            {\color{blue}\httplink[ChainerRL]{github.com/chainer/chainerrl}}
            library.
    \end{itemize}
}
\vspace{0.5em}

% EDUCATION ------------------------------------------------------------
\section{Education}
\cventry{Sep 2016 -- Sep 2018}
{M.Eng., Major in Information Architecture}
{Waseda University}
{Fukuoka, JP}{}
{Thesis: Autoencoder-based Feature Extraction Techniques for Protein
    Function Prediction}
\vspace{0.5em}

\cventry{Jun 2011 -- Jun 2016}{B.S., Electronics \& Communications Engineering}
{Ateneo de Manila University}
{Metro Manila, PH}
{\textit{Cum Laude}}
{
    Thesis: Appliance Recognition using Hall-Effect Current Sensors for
    Power Management Systems\\
    Minor in Philosophy
}
\vspace{0.5em}


\subsection{Fellowships}

\cventry{Jan 2018}
{Fellow}
{RIKEN-Advanced Institute for Computational Sciences}
{Kobe, JP}{RIKEN International School for Data Assimilation}
{Studied data assimilation techniques (3DVar, Kalman Filters, etc.) for real-time numerical simulations.}
\vspace{0.5em}

\cventry{Sep 2015 -- Jan 2016}
{Exchange Student, Fall Semester}
{Institut Catholique d'Arts et M\'etiers}
{Lille, FR}{}
{Took courses in control systems and software development}

% PROJECTS -------------------------------------------------------------
\section{Open-source Software}

I've maintained several open-source projects in the scientific tooling space. 
You can also visit my {\color{blue}\httplink[Github profile]{github.com/ljvmiranda921}} for more information.\\


\cventry{2023}
{\color{blue}
    \httplink[ljvmiranda921/calamanCy]{github.com/ljvmiranda921/calamanCy}}
{calamanCy}{}{}
{
    A natural language processing toolkit for building Tagalog pipelines based on spaCy and written on Python.
}

\cventry{2021}
{\color{blue}
    \httplink[explosion/spaCy]{github.com/explosion/spaCy}}
{spaCy}{}{}
{
    An industrial-strength natural language processing (NLP) software. 
    I'm one of the core contributors as part of the spaCy team.
    I also contributed to related software such as spacy-llm and spaCy projects.
}

\cventry{2017}
{\color{blue} \httplink[ljvmiranda921/pyswarms]{github.com/ljvmiranda921/pyswarms}}
{PySwarms}{}{}
{
    A Python-based framework for implementing swarm optimization
    algorithms. Software paper was published in the \textit{Journal of Open
        Source Software} (JOSS).
}


% AWARDS AND CERTIFICATIONS ----------------------------------------------------------
\section{Awards and Certifications}

\subsection{Professional Certifications}
\cvlanguage{}{Google Cloud Professional Data Engineer (Certification ID: enjfUz)}{2018}

\subsection{Scholarships}
\cvlanguage{}{Monbugakusho (MEXT) Japanese Government Scholarship}{2016}
\cvlanguage{}{French Ministry of Foreign and European Affairs Grant}{2015}
\cvlanguage{}{Department of Science \& Technology SEI Merit Scholarship}{2011}
\cvlanguage{}{Ateneo College Scholarship}{2011}


% BIBLIOGRAPHY ---------------------------------------------------------
\nocite{*}
\bibliographystyle{unsrt}
\bibliography{publications}

\end{document}
