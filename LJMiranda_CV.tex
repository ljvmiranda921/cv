%% start of file `template.tex'.
%% Copyright 2006-2015 Xavier Danaux (xdanaux@gmail.com).
%
% This work may be distributed and/or modified under the
% conditions of the LaTeX Project Public License version 1.3c,
% available at http://www.latex-project.org/lppl/.


\documentclass[11pt,a4paper,sans]{moderncv}


% moderncv themes
\moderncvstyle{banking}
\moderncvcolor{black}

% character encoding
\usepackage[utf8]{inputenc}                     % set encoding
\usepackage[scale=0.75]{geometry}               % customize paper
\usepackage[unicode]{hyperref}                  % to use hyperlinks
\usepackage{xcolor}                             % better colors
\usepackage{color}                              % syntactic sugar

% Setup bib formatting
\makeatletter\renewcommand*{\bibliographyitemlabel}{\@biblabel{\arabic{enumiv}}}\makeatother

% Setup fonts
\renewcommand*{\namefont}{\fontsize{24}{29}\mdseries\upshape}
\renewcommand{\familydefault}{\sfdefault}

% Setup bullet points
\renewcommand{\labelitemi}{\scriptsize\color{black}{$\bullet$}}


% PERSONAL DATA --------------------------------------------------------
\firstname{Lester James V.}
\familyname{Miranda}
\address{Blk 12 Lot 16 Palmera Northwinds I, City of San Jose del Monte
Bulacan, Philippines}
\email{ljvmiranda@gmail.com}
\homepage{ljvmiranda921.github.io}
\social[github]{ljvmiranda921}
\extrainfo{Last update: \today}

\begin{document}

\maketitle

% EXPERIENCE -----------------------------------------------------------
\section{Experience}
\cventry{Oct 2018 -- Present}
        {Machine Learning Researcher}
        {Thinking Machines Data Science}
        {Taguig, Philippines}{Machine Learning Team}
        {
        }

% INTERNSHIPS -----------------------------------------------------------
\subsection{Internships}

\cventry{Aug 2018 -- Sep 2018}
        {Research Intern}
        {Preferred Networks, Inc.}
        {Tokyo, Japan}{ChainerRL Team}
        {Implemented parallelization (Batch Proximal Policy Optimization) to
        the ChainerRL library.}

\cventry{Apr 2015 -- May 2015}
        {Intern}
        {Manila Electric Company}
        {Pasig, Philippines}{Strategy, Architecture \& Governance Team}
        {Investigated business use-cases for 3rd IT Computing platforms (big data,
        mobility, drone automation).}


% EDUCATION ------------------------------------------------------------
\section{Education}
\cventry{Sep 2016-- Sep 2018}
        {M.Eng., Major in Information Architecture}
        {Waseda University}
        {Fukuoka, Japan}{}
        {Thesis: Autoencoder-based Feature Extraction Techniques for Protein
        Function Prediction}

\cventry{Jun 2011-- Jun 2016}{B.S., Electronics \& Communications Engineering}
        {Ateneo de Manila University}
        {Quezon City, Philippines}
        {\textit{Cum Laude}}
        {Minor in Philosophy}

\cventry{Sep 2015 -- Jan 2016}
        {Exchange Student, Fall Semester}
        {Institut Catholique d'Arts et M\'etiers}
        {Lille, France}{}
        {}

\subsection{Fellowships}

\cventry{Jan 2018}
        {Participant}
        {RIKEN-Center for Computational Sciences}
        {Kobe, Japan}{RIKEN International School for Data Assimilation}
        {Learned data assimilation techniques such as 3DVar, Kalman
        Filters for real-time numerical simulations.}

% PROJECTS -------------------------------------------------------------
\section{Selected Projects}

\subsection{Open-Source Software}

\cventry{2018}
        {\color{blue} \httplink[ljvmiranda921/gym-lattice]{github.com/ljvmiranda921/gym-lattice}}
        {Gym-Lattice}{}{}
        {
        A Python library that provides a reinforcement learning environment
        to solve the protein folding problem. Creates an OpenAI gym-like
        interface of the HP-Lattice structure in statistical mechanics.
        }

\cventry{2017}
        {\color{blue} \httplink[ljvmiranda921/pyswarms]{github.com/ljvmiranda921/pyswarms}}
        {PySwarms}{}{}
        {
        A Python-based framework for implementing swarm optimization
        algorithms. Features include visualization and hyperparameter
        optimization. Software paper was published in the Journal of Open
        Source Software (JOSS).
        }

\cventry{2017}
        {\color{blue} \httplink[scikit-multilearn]{github.com/scikit-multilearn/scikit-multilearn}}
        {Scikit-Multilearn}{}{}
        {
        A Python library that provides a \texttt{scikit-learn} interface for
        multilabel classification. Role: Collaborator. Responsibilities include
        resolving issues and updating library documentation.
        }

% SCHOLARSHIP ----------------------------------------------------------
\section{Scholarships Received}
\cvlanguage{}{Monbugakusho (MEXT) Japanese Government Scholarship}{2016}
\cvlanguage{}{French Ministry of Foreign and European Affairs Grant}{2015}
\cvlanguage{}{Department of Science \& Technology SEI Merit Scholarship}{2011}
\cvlanguage{}{Ateneo College Scholarship, 100 Tuition and Fees}{2011}
% BIBLIOGRAPHY ---------------------------------------------------------
\nocite{*}
\bibliographystyle{ieeetr}
\bibliography{publications}

\end{document}
