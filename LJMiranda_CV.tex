%% start of file `template.tex'.
%% Copyright 2006-2015 Xavier Danaux (xdanaux@gmail.com).
%
% This work may be distributed and/or modified under the
% conditions of the LaTeX Project Public License version 1.3c,
% available at http://www.latex-project.org/lppl/.


\documentclass[12pt,a4paper]{moderncv}


% moderncv themes
\moderncvstyle{banking}
\moderncvcolor{black}

% character encoding
\usepackage[utf8]{inputenc}                     % set encoding
\usepackage[scale=0.75]{geometry}               % customize paper
\usepackage[unicode]{hyperref}                  % to use hyperlinks
\usepackage{xcolor}                             % better colors
\usepackage{color}                              % syntactic sugar
% Font settings
\usepackage[T1]{fontenc}
% Serif body font
% \usepackage{charter}
\usepackage{libertine}
% Math font
%\usepackage{libertinust1math}
% \usepackage[charter]{mathdesign}
% Monospaced font
\usepackage{sourcecodepro}
% Sans-serif font
%\usepackage[lf]{FiraSans}

% Setup bib formatting
\makeatletter\renewcommand*{\bibliographyitemlabel}{\@biblabel{\arabic{enumiv}}}\makeatother

% Setup fonts
\renewcommand*{\namefont}{\fontsize{24}{29}\mdseries\upshape}
%\renewcommand{\familydefault}{\sfdefault}

% Setup bullet points
\renewcommand{\labelitemi}{\scriptsize\color{black}{$\bullet$}}


% PERSONAL DATA --------------------------------------------------------
\firstname{Lester James V.}
\familyname{Miranda}
\address{Metro Manila, Philippines}
\email{ljvmiranda@gmail.com}
%\mobile{(+63)905-258-1624}
\homepage{ljvmiranda921.github.io}
\social[github]{ljvmiranda921}
\social[twitter]{ljvmiranda921}
\extrainfo{Last updated: \today}

\title{CV}

\begin{document}

\maketitle


% EXPERIENCE -----------------------------------------------------------
\section{Experience}

\cventry{Oct 2021 -- present}
{Machine Learning Engineer}
{Explosion}
{Germany}{spaCy Team (Remote)}
{
    \begin{itemize}
        \item Responsibilities include: development of {\color{blue}
            \httplink[spaCy]{spacy.io}} and associated open-source
            projects, creation of course content and educational materials, and
            development of internal projects and tooling. 
    \end{itemize}
}

\cventry{Oct 2018 -- Jul 2021}
{Machine Learning Researcher}
{Thinking Machines Data Science}
{Philippines}{Machine Learning Team}
{
    \vspace{3px}
    \textit{Client Projects}
    \begin{itemize}
        \item Developed several production-grade machine learning applications
            for a major investment firm in Singapore (Feb 2019 -- Jul 2021): 
            \begin{itemize}
                \item A semantic search engine for an in-house mobile
                    application based on natural language processing and
                    Transformer (DistilBERT, RoBERTa) techniques. 
                \item A document processing tool powered by natural language
                    processing and optical character recognition (OCR) for an
                    in-house application to automate report management.
                \item An industry classifier using natural language processing
                    and multilabel / hierarchical classification to
                    streamline portfolio risk assessment. 
            \end{itemize}
        \item As Tech Lead, mentored a project team to deliver a {\color{blue}
            \httplink[large-scale
            digitization project]{stories.thinkingmachin.es/doc-ai-world-bank-case-study/}} of all LGU financial statements across the
            country for The World Bank.  (Jul 2020 -- Sep 2020)
        \item Applied computer vision techniques such as Mask R-CNN and Faster
        R-CNN to automate {\color{blue}\httplink[building detection from aerial
        images]{stories.thinkingmachin.es/wealth-detection-satellite-image/}}
        for one of the largest telecommunications companies in the Philippines.
        (Oct 2018 -- Feb 2019)
    \end{itemize}
    \vspace{3px}
    \textit{Internal Efforts}
    \begin{itemize}
        \item As Team Lead, managed a team of six in various research and
            engineering efforts in the document processing space for the
            {\color{blue}\httplink[DocumentAI]{thinkingmachin.es/solutions/\#documentIntelligence}}
            product. (Apr 2020 -- Jun 2021)
    \end{itemize}
}

% INTERNSHIPS -----------------------------------------------------------
\subsection{Internships}
\cventry{Aug 2018 -- Sep 2018}
{Research Intern}
{Preferred Networks, Inc.}
{Japan}{ChainerRL Team}
{
    \begin{itemize}
        \item Developed a reinforcement learning parallelization framework
            based on batch Proximal Policy Optimization (PPO) for the
            open-source
            {\color{blue}\httplink[ChainerRL]{github.com/chainer/chainerrl}}
            library.
    \end{itemize}
}

\cventry{Apr 2015 -- May 2015}
{Intern}
{Manila Electric Company}
{Philippines}{Strategy, Architecture \& Governance Team}
{
    \begin{itemize}
        \item Investigated use-cases for IT Computing platforms: big
              data, mobility, \& drone automation.
    \end{itemize}
}


% EDUCATION ------------------------------------------------------------
\section{Education}
\cventry{Sep 2016 -- Sep 2018}
{M.Eng., Major in Information Architecture}
{Waseda University}
{Japan}{}
{Thesis: Autoencoder-based Feature Extraction Techniques for Protein
    Function Prediction}

\cventry{Jun 2011 -- Jun 2016}{B.S., Electronics \& Communications Engineering}
{Ateneo de Manila University}
{Philippines}
{\textit{Cum Laude}}
{
    Thesis: Appliance Recognition using Hall-Effect Current Sensors for
    Power Management Systems\\
    Minor in Philosophy
}


\subsection{Fellowships \& Others}

\cventry{Jan 2018}
{Participant}
{RIKEN-Advanced Institute for Computational Sciences}
{Japan}{RIKEN International School for Data Assimilation}
{Studied data assimilation techniques (3DVar, Kalman Filters, etc.) for real-time numerical simulations.}

\cventry{Sep 2015 -- Jan 2016}
{Exchange Student, Fall Semester}
{Institut Catholique d'Arts et M\'etiers}
{France}{}
{Took courses in control systems and software development}

% PROJECTS -------------------------------------------------------------
\section{Selected Projects}

\subsection{Open-Source Software}


\cventry{2019}
{\color{blue}
    \httplink[thinkingmachines/geomancer]{github.com/thinkingmachines/geomancer}}
{Geomancer}{}{}
{
    A Python library to automate feature engineering of geospatial data. Uses
    OpenStreetMap (OSM) and BigQuery. Developed at Thinking Machines.
}

\cventry{2018}
{\color{blue} \httplink[ljvmiranda921/gym-lattice]{github.com/ljvmiranda921/gym-lattice}}
{Gym-Lattice}{}{}
{
    A reinforcement learning environment for the protein folding problem based
    on the HP-lattice structure.
}

\cventry{2017}
{\color{blue} \httplink[ljvmiranda921/pyswarms]{github.com/ljvmiranda921/pyswarms}}
{PySwarms}{}{}
{
    A Python-based framework for implementing swarm optimization
    algorithms. Software paper was published in the \textit{Journal of Open
        Source Software} (JOSS).
}


% AWARDS AND CERTIFICATIONS ----------------------------------------------------------
\section{Awards and Certifications}

\subsection{Professional Certifications}
\cvlanguage{}{Google Cloud Professional Data Engineer (Certification ID: enjfUz)}{2018}

\subsection{Scholarships}
\cvlanguage{}{Monbugakusho (MEXT) Japanese Government Scholarship}{2016}
\cvlanguage{}{French Ministry of Foreign and European Affairs Grant}{2015}
\cvlanguage{}{Department of Science \& Technology SEI Merit Scholarship}{2011}
\cvlanguage{}{Ateneo College Scholarship, 100\% Tuition and Fees (100TF)}{2011}


% BIBLIOGRAPHY ---------------------------------------------------------
\nocite{*}
\bibliographystyle{ieeetr}
\bibliography{publications}

\end{document}
